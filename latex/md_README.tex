
\begin{DoxyItemize}
\item Задание для летней практики 2017
\end{DoxyItemize}

Приложение -\/ консольное.

Чтобы запустить программу, необходимо написать следующие команды в терминале\+: 
\begin{DoxyCode}
git clone https://github.com/moskanka/DataMining
./get-reestr-mtmc.sh
\end{DoxyCode}


В процессе работы приложение заходит на \href{https://reestr.minsvyaz.ru/reestr/}{\tt сайт Единого реестра российских программ для ЭВМ и БД} и собирает оттуда необходимую информацию. Результаты записываются в файл {\ttfamily reestr.\+csv}, а так же выдаются прямо в консоль в {\ttfamily csv}-\/формате

Кодировка {\ttfamily csv}-\/файла\+: U\+T\+F-\/8 без B\+OM.

Ошибки выдаются в отдельном потоке -\/ {\ttfamily stderror}

{\bfseries Пример работы программы для одной из программ реестра\+:}


\begin{DoxyItemize}
\item Страница продукта\+: 
\item Выходной {\ttfamily csv}-\/файл\+: 
\item Если во входных данных нет искомого поля, то в него ставится значение по умолчанию, а если оно не оговоренно, то поле оставляется пустым.
\end{DoxyItemize}

{\bfseries Документация функций и классов\+:}

Чтобы открыть документацию к этому приложению необходимо написать в терминале (находясь в корневой папке проекта)\+: 
\begin{DoxyCode}
doxygen Doxyfile
cd html
gnome-open index.html
\end{DoxyCode}
 Так же можно вручную открыть файл {\ttfamily index.\+html} в удобном вам браузере. 